\documentclass[11pt, a4paper]{article} % Font size
\usepackage{float} % To impose the position of the figure at desired place.
\usepackage{amsmath, amsfonts, amsthm} % Math packages
\usepackage{hyperref} % To hiperlink things on the internet.
\usepackage{cite} % To cite stuff
\usepackage{listings} % Code listings, with syntax highlighting
\usepackage[english]{babel} % English language hyphenation
\usepackage{graphicx} % Required for inserting images
\usepackage{sectsty} % Allows customising section commands
\usepackage{booktabs} % Required for better horizontal rules in tables
\usepackage{enumitem} % Required for list customisation
\usepackage{geometry} % Required for adjusting page dimensions and margins
\usepackage{svgcolor}
\usepackage{svg}
\usepackage{multirow}
\usepackage{comment} % To remove big parts of text.
\usepackage[utf8]{inputenc} % Required for inputting international characters
\usepackage[T1]{fontenc} % Use 8-bit encoding
\usepackage{fourier} % Use the Adobe Utopia font for the document
\usepackage[nottoc]{tocbibind}
\usepackage{cancel} % To cross and cancel values
\usepackage{bigints} % In case you need bigger integrants
\usepackage{xcolor} %To use colors
\usepackage{subcaption}
\usepackage[skip=10pt plus1pt, indent=0pt]{parskip}
\newcommand{\ket}[1]{\vert #1 \rangle}
\newcommand{\bra}[1]{\langle #1 \vert}
\renewcommand{\it}{\textit}
\definecolor{Burgundy}{RGB}{155,5,8}
\hypersetup{
    colorlinks=true,
    linkcolor= Burgundy,
	citecolor = teal,
    filecolor=magenta,      
    urlcolor=Burgundy,
    }



\geometry{
	paper=a4paper, % Paper size, change to letterpaper for US letter size
	top=1.5cm, % Top margin
	bottom=3cm, % Bottom margin
	left=3cm, % Left margin
	right=3cm, % Right margin
	headheight=0.5cm, % Header height
	footskip=1.5cm, % Space from the bottom margin to the baseline of the footer
	headsep=0.75cm, % Space from the top margin to the baseline of the header
	%showframe, % Uncomment to show how the type block is set on the page
}




\setlength\parindent{0pt} % Removes all indentation from paragraphs
\graphicspath{{Figures/}{./}}% Specifies where to look for included images (trailing slash required)
%\setlist{noitemsep} % No spacing between list items
\sectionfont{\vspace{6pt}\centering\normalfont\scshape} % \section{} styling
\subsectionfont{\normalfont\bfseries} % \subsection{} styling
\subsubsectionfont{\normalfont\itshape} % \subsubsection{} styling
\paragraphfont{\normalfont\scshape} % \paragraph{} styling





%--------------------------------------------------------------------------------
%	TITLE SECTION
%--------------------------------------------------------------------------------

\title{	
	\normalfont\normalsize
	\textsc{}\\ % Your university, school and/or department name(s)
	\vspace{5pt} % Whitespace
	\rule{\linewidth}{0.2pt}\\ % Thin top horizontal rule
	\vspace{10pt} % Whitespace
	{\huge Cosmological Inflation}\\ % The assignment title
	\vspace{5pt} % Whitespace
	\rule{\linewidth}{-2pt}\\ % Thick bottom horizontal rule
	\vspace{-25pt} % Whitespace
	\date{}
}
\author{}





\begin{document}
\maketitle 



	%\begin{figure}[ht]\label{fig: something}
     %   \centering
      %  \includegraphics[scale=0.7]{test}
       % \caption{caption}
      %\end{figure}

\section*{Flashback of better times}

In the previous \href{https://panopepino.github.io/web_page/downloads/notes/cosmo_classical.pdf}{classical cosmology notes}, we have extensively discussed the set of classical equations that provide an accurated description of the dynamics of the Universe.

\textcolor{red}{TO DO:}

Mention somehow that this is a classical approach and one has to go to Q regimes to fully appreciate the power of perturbations that generate late time structures.

\begin{enumerate}
	\item Describe that the model works fine for late time cosmo and the hot big bang works fine until some regime.
	\item Elaborate on energy scales, down to inflation regime.
	\item Elaborate on regular cosmo problems, with equations and plots.
	\item Introduce inflation, how it solves problems described before.
	\item Present EOM for inflation, parameters and values.
	\item talk a little about different potentials.
	\item Mention some quintessence. Connection to UV complete theories?
	\item Contact with observations. Prove?
\end{enumerate}


\cite{Danielsson:2018ztv}





\bibliography{biblio}
\bibliographystyle{ieeetr}


\end{document}