\documentclass[11pt, a4paper]{article} % Font size
\usepackage{float} % To impose the position of the figure at desired place.
\usepackage{amsmath, amsfonts, amsthm} % Math packages
\usepackage{hyperref} % To hiperlink things on the internet.
\usepackage{cite} % To cite stuff
\usepackage{listings} % Code listings, with syntax highlighting
\usepackage[english]{babel} % English language hyphenation
\usepackage{graphicx} % Required for inserting images
\usepackage{sectsty} % Allows customising section commands
\usepackage{booktabs} % Required for better horizontal rules in tables
\usepackage{enumitem} % Required for list customisation
\usepackage{geometry} % Required for adjusting page dimensions and margins
\usepackage{comment} % To remove big parts of text.
\usepackage[utf8]{inputenc} % Required for inputting international characters
\usepackage[T1]{fontenc} % Use 8-bit encoding
\usepackage{fourier} % Use the Adobe Utopia font for the document
\usepackage[nottoc]{tocbibind}
\usepackage{cancel} % To cross and cancel values
\usepackage{bigints} % In case you need bigger integrants
\usepackage{xcolor} %To use colors
\usepackage[skip=10pt plus1pt, indent=0pt]{parskip}
\newcommand{\ket}[1]{\vert #1 \rangle}
\newcommand{\bra}[1]{\langle #1 \vert}
\renewcommand{\it}{\textit}


\geometry{
	paper=a4paper, % Paper size, change to letterpaper for US letter size
	top=1.5cm, % Top margin
	bottom=3cm, % Bottom margin
	left=3cm, % Left margin
	right=3cm, % Right margin
	headheight=0.5cm, % Header height
	footskip=1.5cm, % Space from the bottom margin to the baseline of the footer
	headsep=0.75cm, % Space from the top margin to the baseline of the header
	%showframe, % Uncomment to show how the type block is set on the page
}




\setlength\parindent{0pt} % Removes all indentation from paragraphs
\graphicspath{{Figures/}{./}}% Specifies where to look for included images (trailing slash required)
%\setlist{noitemsep} % No spacing between list items
\sectionfont{\vspace{6pt}\centering\normalfont\scshape} % \section{} styling
\subsectionfont{\normalfont\bfseries} % \subsection{} styling
\subsubsectionfont{\normalfont\itshape} % \subsubsection{} styling
\paragraphfont{\normalfont\scshape} % \paragraph{} styling







%--------------------------------------------------------------------------------
%	TITLE SECTION
%--------------------------------------------------------------------------------

\title{	
	\normalfont\normalsize
	\textsc{}\\ % Your university, school and/or department name(s)
	\vspace{5pt} % Whitespace
	\rule{\linewidth}{0.2pt}\\ % Thin top horizontal rule
	\vspace{10pt} % Whitespace
	{\huge Classical Cosmology}\\ % The assignment title
	\vspace{5pt} % Whitespace
	\rule{\linewidth}{-2pt}\\ % Thick bottom horizontal rule
	\vspace{-25pt} % Whitespace
	\date{}
}
\author{}





\begin{document}
\maketitle 



	%\begin{figure}[ht]\label{fig: something}
     %   \centering
      %  \includegraphics[scale=0.7]{test}
       % \caption{caption}
      %\end{figure}

\subsection*{Keywords}
\begin{itemize}
	\item Expansion of the Universe
	\item Dark Energy
	\item General Relativity
	\item Cosmological Constant
	\item Measure of cancel
	\item de Sitter Space
	\item Friedmann equations
\end{itemize}

\subsection*{What do we know about the Universe?}

Several satellites that we have sent there, outside\footnote{Plus all evidence collected also from earth surface.}, to the empty and cold darkness of space, have provided enough data to prove that any galaxy far away (and not so far away) from us is in a process of getting further away. On top of that, increasing their distance today even faster than yesterday. It could perhaps be due to the fact that the Universe itself sees human beings as a potential plague, and want to avoid us. Or could be because some sort of colossal multidimensional being decided to streach the fabric of spacetime itself, just for fun.

Nonsense apart, nowadays, the most accepted reason why the Universe is pushed apart in an accelerated manner is \textit{Dark Energy}. Ah, wonderful. And what is \textit{Dark Energy}? To explain this is still an open problem\footnote{A really big one. Or a small one?}\footnote{As one could say in Spanish, \textit{"Un problema nimio"}. Nimio is an adjective that can be used for both small and big.}. But let us first step back more than a century ago, to understand the synthesis of our current understanding of Cosmology.


In 1915, good old Einstein published his theory of \textit{General Relativity} (GR) \cite{}. The world did not become a better place due to this, but at least we were provided with a ridiculous powerful tool to describe low-energy gravitational events. In the following years after the publication, Friedmann, Hubble, Lema\^itre \cite{} (among many others) used GR technology to describe the Universe as a whole. Their work set the foundations of what we call today \textit{The Standard Model of Cosmology} \cite{}. This model is good enough to describe our current observations of the Universe. And not only that. If we reversed the observed expansion back to very close the beginning of everything, we could still have a really good description of the events happening in the almost newborn Universe. It has 3 really simple foundations:

\begin{enumerate}
	\item \textit{Copernican}: Our planet occupies not special position in the Universe.
	\item \textit{GR + Expansion}: Einstein equations describe gravitational dynamics with accuracy at low-energy physics and Hubble's discovery (The expansion of the Universe) in 1929 is correct.
	\item \textit{Perfect fluidity}: We can assume that all contents in the Universe behave as a perfect fluid.
\end{enumerate}

Of course this model has its flaws, but we leave these downsides for future lines.

At really big scales, Copernican principle holds. No point in space occupies a spatial position. Wherever you sit at and look at, everything will be more or less the same. In technical words, this implies \textit{homogeneity} and \textit{isotropy}. Good, seems simple. Next step is to find a reliable way to measure distance, hence to be able to describe the geometry of spacetime. This is given by the \textit{line invariant}, which takes the famous \textit{FRLW} form, adequate to describe a Lorentzian signature spacetime with a high degree of symmetry as the one we seem to live in. This can be written as:


\begin{equation}\label{eq: line_invariant_FLRW}
	ds^{2} = - N^{2}(t) dt^{2} + a(t)^2 \left(\frac{dr^{2}}{1- k r^{2}}+ r^{2} \left(d\theta^{2} + \sin[\theta]^{2} d\phi^{2}\right)\right).
\end{equation}
Where $N(t)$ is a lapse function and $a(t)$ is the scale factor that describes the expansion (or contraction) of $3D$ spacial slices. Before we start talking about the spatial properties of previous line invariant (\ref{eq: line_invariant_FLRW}), let us first discuss about the lapse function $N(t)$. 

This function is in charge of time reparametrization invariance. As we do not want to overcomplicate our computation, the most useful and convienent choices for $N(t)$ are:

\begin{itemize}
	\item $N(t) = 1$: This is the choice of \textit{global time}. \textcolor{red}{Add some explanation}
	\item $N(t) = a(t)$: This is the so-called \textit{conformal time} gauge choice. This choice allows the observer to co-move with the expansion of the universe, even in the time coordinate, accounting for any funky effect... \textcolor{red}{Add some explanation}
\end{itemize}

The extra parameter we have not talked about yet is $k$. This has the power to change curvature of space. It comes in three different flavours:

\begin{itemize}
	\item $k = 0$: A quite boring case. No curvature, where spatial sections of our geometry are \it{flat}, like $\mathbb{R}^{3}$.
	\item $k = 1$: With this value, spatial sections are \it{closed}, like in $\mathbb{S}^{3}$.
	\item $k = -1$: Spatial sections are \it{open}, as in the hyperboloid $\mathbb{H}^{3}$.
\end{itemize}

So the line invariant (\ref{eq: line_invariant_FLRW}) allows us to describe a dynamical universe with different types of spatial curvature. \textcolor{red}{Continue here. This is taking more time than expected.}




\cite{Danielsson:2018ztv}





\bibliography{biblio}
\bibliographystyle{ieeetr}


\end{document}