\section{Dark bubbles from string theory: An explicit construction}\label{sec: db_from_st}
\textcolor{red}{I GUESS THAT HERE I NEED TO WRITE A GOOD SUMMARY OF THE MODEL I WILL USE (OSCAR'S). SPECIALLY WHAT IT WAS USED FOR IN THE PAST AND HOW GOOD IT MAY CONNECT TO THE KIND OF SCENARIO WE HAD IN MIND FOR BUBBLES.}

\textcolor{red}{EMPHASISE THAT THE STACK OF BLACK BRANES IS A BLACK HOLE IN HIGHER DIM.}
\subsection{A black hole in the bulk: Ten-dimensional Kerr or five-dimensional Reissner-Nordström one?}
Let us start by defining the parametric coordinates that will be used to describe the 10 dimensional rotating background\footnote{It is important to notice the choice of $z$ to describe the $\text{AdS}_{5}$ throat direction. Spoiler alert: There will be a convenient change of coordinates soon to adequate $r$-direction. Stay alert.}:
\textcolor{red}{OBSERVE THE CHANGE OF COORDINATE DEFINITION WRT THE PAPER. SO THE RESULTING EQUATIONS ARE MORE FAMILIAR WITH THE USUAL THROAT.}
\begin{equation}\label{eq: 10D_coordinates}
	x^{\mu} = \{t, \alpha, \beta, \gamma, z, \theta, \psi, \phi_{1},\phi_{2},\phi_{3}\},
\end{equation}
and its line invariant
\begin{equation}\label{eq: 10Dmetric}
  \text{ds}_{10}^2= \text{ds}_5^2 + L^2 \sum_{i=1}^3 \left\{ \dpar\sigma_i^2 + \sigma_i^2 \left( \dpar\phi_i + \tfrac{1}{L} A(z) \right)^2 \right\},
\end{equation}
where $A(z)$ is a 10 dimensional one-form that will be later discussed and the $\sigma_i$-functions are combinations of trigonometric functions of two angles\footnote{\textcolor{red}{ADD REF AND COMMENTS ABOUT THE PROCEDENCE OF THIS ANGLES (from where to where do they run) AND PROBABLY, COMMENTS ABOUT THE SPHERE?}} $\{\theta, \psi$\} of the $S^{5}$ as:
\begin{equation}
 \sigma_1 = \sin\theta \ , \qquad \sigma_2 = \cos\theta \sin\psi \ , \qquad \sigma_3 = \cos\theta \cos\psi \ .
\end{equation}
As introduced in section (\ref{sec: simple_example}), $L$ is the $\text{AdS}_{5}$ radius in Eq. (\ref{eq: 10Dmetric}). It is important to notice that this radius also sets the scale of the extra compact dimensions of $S^{5}$. This can already raise some eyebrows, as it points to lack of scale separation, as discussed in [SEC Landscape]. However, this will be a key signature of the dark bubble embedding into supegravity, as this model acquires the aforementioned scale separation via tension-to-charge ratio instead. We will later explore this method in section (\ref{subsec: energy_scale}). 

Conducting our attention back to the line invariant (\ref{eq: 10Dmetric}), we can define the five dimensional asymptotically-AdS metric $\text{ds}_{5}$ as:
\begin{equation}\label{eq: 5DmetricQ}
  \text{ds}_{5}^{2} = - h(z)^{-2} f(z) \dpar t^{2}
        + h(z) \big[ f(z)^{-1} \dpar z^{2} + z^{2} \dpar \Omega_{3}^{2} \big] \ ,
\end{equation}
where $d\Omega_{3}^2$ is the usual unit metric of the 3-sphere [ref to COSMO SECTION]. The radial functions $h(z)$ and $f(z)$ are expressed as
\begin{equation}\label{eq: h_and_f_bh}
    \begin{split}
        h(z)&=1+\frac{q^{2}}{z^{2}} \ , \\
 f(r) &= 1 - \frac{m}{z^{2}} + \frac{z^{2}}{L^{2}} h(z)^{3}
    \end{split}
\end{equation}
where $m$ and $q$ are respectively the mass and charge of the five dimensional black. Wait? What five-dimensional black hole? Let us then perform a "massage" to Eqs. (\ref{eq: 5DmetricQ}, \ref{eq: h_and_f_bh}) to provide a more familiar black hole-ish line invariant. One can start this task by performing a simple change of coordinates, defining a new radial coordinate $r$ as:
\begin{equation}\label{eq: change_coordinates}
    r^{2} = z^{2}h(z) = q^{2} + z^{2} .
\end{equation}
This change of coordinates will substantialy transform Eq. (\ref{eq: 5DmetricQ}), yielding:
\begin{equation}\label{eq: metricargument}
    \text{ds}_{5}^{2} = -g(r) \dpar t^{2} + g(r)^{-1} \dpar r^{2}+r^{2} \dpar \Omega_{3}^{2}, \qquad \quad g(r) = 1 + k^{2}r^{2}- \frac{2\kappa_{5} M}{r^{2}} + \frac{\kappa_{5}^{2} Q^{2}}{r^{4}}, 
\end{equation}
with 
\begin{align}\label{eq: identifications_MQ}
    M = \frac{1}{2\kappa_5}\left(m + 2q^2\right), &&  Q^2 = \frac{1}{\kappa_5^2}q^2(m+q^2),
\end{align}
with $\kappa_5 = 8\pi G_5$. It is now easy to see that the change of variables described in Eq. (\ref{eq: change_coordinates}) casts the line invariant (\ref{eq: 5DmetricQ}) in the familiar form\footnote{The patch we are describing with the metric (\ref{eq: 5DmetricQ}) corresponds to the requirement $r\geqslant q$. This is actually not an issue since the  horizon $r_{H}$ will be covered by the patch.} of a five dimensional Reissner-Nordström black hole living in an $\text{AdS}$ vacuum [REF]. Observe that, for a small black hole, with $r_{H} \ll L$, we recover a flat space description and it is required $Q<M$ to get a horizon and no naked singularity. On the other hand, if we want to study a horizon larger that the $\text{AdS}_{5}$ radius, this immediately implies $Q\ll M$.

Let us recap now the interpretation of both ten and five dimensional geometries. In the ten dimensional approach, we will observe a Kerr black hole which rotates in three of the compact directions of $S^{5}$ (i.e. $\{\phi_{i}\}$). When we zoom out and not have resolution of the five dimensional sphere, the black hole no longer rotates. It becames "static" from the point of view of a five dimensional observer, but it acquires a charge $q$ due to its motion in the compact directions. This charge becomes effective (i.e. charge $Q$) when the change of coordinates (\ref{eq: change_coordinates}) is performed to recover the more familiar description of a Reissner-Nordström black hole living in $\text{AdS}$.

As it is well known, this type of black holes enjoy two different horizons [REFS to book]: The \textit{outer} horizon (i.e. event horizon) and the \textit{inner} one (i.e. the Cauchy horizon). When these two horizon become degenerate (i.e. $r_{H} = r_{h}$), this corresponds to an extremal black hole description. We will not further discuss charged black holes and their connection to the weak gravity conjecture here (see [SWAMP] for more details), but, for computational purposes in the incoming sections, we find more convenient to express (\ref{eq: metricargument}) in terms of the two aforementioned horizons: $\left\{r_{h}, r_{H}\right\}$ as the inner and outer horizons. In this way, we can express $g(r)$ in Eq. (\ref{eq: metricargument}) with the following Ansatz:
\begin{equation}\label{eq: horizon_ansatz}
    g(r) = \frac{k^{2}}{r^{4}}\left(r^{2} + c\right)\:\left(r^{2} -r_{h}^{2}\right)\:\left(r^{2} - r_{H}^{2}\right),
\end{equation}
with $c \in \mathbb{R}$. A simple match between Eqs. (\ref{eq: metricargument}, \ref{eq: horizon_ansatz}) shows that:
\begin{equation}\label{eq: MQhorizons}
    \begin{split}
        &M= \frac{1}{2 \kappa_{5}}\left[r_h^{2} + r_H^{2} +k^{2} \left( r_{h}^{4}  + r_h^{2} r_H^{2}  + r_H^4 \right)\right],  \\
    &Q^2 =\frac{r_{h}^{2}\:r_{H}^{2}}{\kappa_5^2} \:\left[1 + k^{2}\left(r^{2}_{h}+ r_{H}^{2}\right)\right],\\
    &c = r_{h}^{2} + r_{H}^{2} + \tfrac{1}{k^{2}}.
    \end{split}
\end{equation}
which simplifies when the black hole is extremal. Let us keep in mind these identifications for future section and conduct our attention back to the one-form potential $A(z)$ in Eq. (\ref{eq: 10Dmetric}). This form acts as a gauge field from the 5D point of view \textcolor{red}{WHY?}. In term of the old coordinates (i.e. $\{t, \alpha, \beta, \gamma, z\}$) it can be written as:
\begin{equation}\label{eq: gaugepot_A}
 A(z) = \frac{q}{z_H^2+q^2}\sqrt{ \left(q^2+z_H^2 \right) + \frac{z_H^4}{L^2} h(z_H)^3} \left(1-\frac{z_H^2+q^2}{z^2+q^2}\right) \dpar t.
\end{equation}
Observe that gauge freedom has been summoned to add a constant that sets $A(z_H) = 0$. This is required as a $A(z)$ is temporal gauge potential, and must vanish at the horizon. This potential can be expressed in a more handable way in two steps:
\begin{enumerate}
    \item Part of the argument of the root in Eq. (\ref{eq: gaugepot_A}) is the mass $m$ of the black hole\footnote{This can be easily obtained by computing the outer horizon in Eq. (\ref{eq: 5DmetricQ})}. This is:
    \begin{equation}
        m = z_{H}^{2}\left(1 + \frac{z_{H}^{2}}{L^{2}}h(z_{H})^{3}\right).
    \end{equation}
    \item Furthermore, performing the change of variables (\ref{eq: change_coordinates}) and making use of the identifications (\ref{eq: identifications_MQ}) to further simplify in the new coodinate system, one obtains:
    \begin{equation}\label{eq: gaugepot_A_new_coordinates}
        A(r) = \kappa_{5} Q \left(\frac{1}{r_{H}^{2}} - \frac{1}{r^{2}}\right).
    \end{equation}
\end{enumerate}
With this we finish the black hole description, both from ten and five dimensional perspectives. Let us now explore the explicit embedding of the $D_{3}$-brane in the ten dimensional ambient space.

\subsection{The action of the $D_{3}$-brane}
The rotating D3-branes will source the self dual $F_{5}$ Ramond-Ramond (RR) field strength, which can be written in terms of the radial function $h(r)$ and the one-form $A$ %\cite{Henriksson:2019zph}. What we will need is the corresponding 4-form potential $C_{4}$, with $d\:C_{4}=F_{5}$, given by
\begin{equation}
C_4 = \frac{1}{L}\left[(r^2+q^2)^2 - (r_H^2+q^2)^2\right] dt \wedge \epsilon_3 + L^2 q \sqrt{ r_H^2 + q^2 + \frac{r_H^4}{L^2} h(r_H)^3}\ \sum_i \sigma_i^2\, d\phi_i \wedge \epsilon_3 \ , \label{eq:C4}
\end{equation}
\subsection{Comparing junctions to EOM}
Basically, compare as we did in the paper. Just to prove, that at some specific limit EOM = JC.
\subsection{Collapsing Cosmology: Evacuation required} 
\subsection{Higher curvature corrections to the rescue}
Point to the fact that previous EOM has no CC. Discuss about how to obtain it and elaborate on all pieces in the game. End with the famous $1/N$ corrections that transform in CC. (First part of 5th paper)
\subsection{Energy scales from Dark Bubble embedding: A new hope}\label{subsec: energy_scale}
Compare to observed CC and obtain $N$ to fix all scales. Discuss and defend the size of extra dimensions and the stringy scales.

