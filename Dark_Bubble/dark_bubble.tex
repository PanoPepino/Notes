\documentclass[11pt, a4paper]{article} % Font size
\usepackage{float} % To impose the position of the figure at desired place.
\usepackage{amsmath, amsfonts, amsthm} % Math packages
\usepackage{hyperref} % To hiperlink things on the internet.
\usepackage{cite} % To cite stuff
\usepackage{listings} % Code listings, with syntax highlighting
\usepackage[english]{babel} % English language hyphenation
\usepackage{graphicx} % Required for inserting images
\usepackage{sectsty} % Allows customising section commands
\usepackage{booktabs} % Required for better horizontal rules in tables
\usepackage{enumitem} % Required for list customisation
\usepackage{geometry} % Required for adjusting page dimensions and margins
\usepackage{svgcolor}
\usepackage{svg}
\usepackage{multirow}
\usepackage{comment} % To remove big parts of text.
\usepackage[utf8]{inputenc} % Required for inputting international characters
\usepackage[T1]{fontenc} % Use 8-bit encoding
\usepackage{fourier} % Use the Adobe Utopia font for the document
\usepackage[nottoc]{tocbibind}
\usepackage{cancel} % To cross and cancel values
\usepackage{bigints} % In case you need bigger integrants
\usepackage{xcolor} %To use colors
\usepackage{subcaption}
\usepackage[skip=10pt plus1pt, indent=0pt]{parskip}
\newcommand{\ket}[1]{\vert #1 \rangle}
\newcommand{\bra}[1]{\langle #1 \vert}
\renewcommand{\it}{\textit}
\definecolor{Burgundy}{RGB}{155,5,8}
\hypersetup{
    colorlinks=true,
    linkcolor= black,
	citecolor = Burgundy,
    filecolor=magenta,      
    urlcolor=Burgundy,
    }



\geometry{
	paper=a4paper, % Paper size, change to letterpaper for US letter size
	top=2.5cm, % Top margin
	bottom=2.5cm, % Bottom margin
	left=2.5cm, % Left margin
	right=2.5cm, % Right margin
	headheight=0.5cm, % Header height
	footskip=1.5cm, % Space from the bottom margin to the baseline of the footer
	headsep=0.75cm, % Space from the top margin to the baseline of the header
	%showframe, % Uncomment to show how the type block is set on the page
}




\setlength\parindent{0pt} % Removes all indentation from paragraphs
\graphicspath{{Figures/}{./}}% Specifies where to look for included images (trailing slash required)
%\setlist{noitemsep} % No spacing between list items
\sectionfont{\vspace{6pt}\centering\normalfont\scshape} % \section{} styling
\subsectionfont{\normalfont\bfseries} % \subsection{} styling
\subsubsectionfont{\normalfont\itshape} % \subsubsection{} styling
\paragraphfont{\normalfont\scshape} % \paragraph{} styling










%--------------------------------------------------------------------------------
%	TITLE SECTION
%--------------------------------------------------------------------------------

\title{	
	\normalfont\normalsize
	\textsc{}\\ % Your university, school and/or department name(s)
	\vspace{5pt} % Whitespace
	\rule{\linewidth}{0.2pt}\\ % Thin top horizontal rule
	\vspace{10pt} % Whitespace
	{\huge About Dark Bubbles}\\ % The assignment title
	\vspace{5pt} % Whitespace
	\rule{\linewidth}{-2pt}\\ % Thick bottom horizontal rule
	\vspace{-25pt} % Whitespace
	\date{}
}
\author{}





\begin{document}
\maketitle 
\tableofcontents



	%\begin{figure}[ht]\label{fig: something}
     %   \centering
      %  \includegraphics[scale=0.7]{test}
       % \caption{caption}
      %\end{figure}

\newpage
\section{How to induce dark energy on a bubble: A simple example}
\begin{itemize}
	\item Construct the 5D bubble from basic principles. Refer to appendix/chapters where I talk about RS, non-susy, advanced differential geometry and instantons.
	\item Classic construction. define things static first and then move to radial expansion. Induce positive cosmological constant.
	\item This can perhaps be a good section to talk about Unruh stuff?
\end{itemize}

Let us just start with the most basic set-up: A non-supersymmetric Anti-de Sitter vacuum, with cosmological constant $\Lambda_{D} = - \tfrac{1}{2}(D-1)(D-2)L^{3-D}$. The parameter $L$ represents the curvature radius of the $\text{AdS}_{D}$ vacuum. For computational convenience, it will be easier to refer to this parameter with its inverse, the vacuum scale $k = \tfrac{1}{L}$.

As we discussed in section [NON-SUSY], the aforementioned $\text{AdS}_{D}$ vacuum is unstable and will eventually decay to a more metastable solution, via $D_{p}$-brane nucleation. This $D_{p}$-brane, which is a Coleman-de Luccia/ BT instanton [add ref to CdL and or section], will mediate such decay, separating two different $\text{AdS}_{D}$ vacua. From now on, we will refer to the vacua living inside the hyper-volume encoded by the $D_{p}$-brane as the \textit{inside} vacuum, denoted by a minus (-) subscript sign. The tandem brane-inside will be colloquially called \textit{Bubble}. On the other hand, the region \textit{outside} the bubble has not yet decayed, and will be denoted by a positive (+) subscript sign.

Finally, as the aim of this thesis is to discuss phenomenological aspects in four dimensions\footnote{Other bubble dimensionalities have been explored in [IVANO STUFF]}, and given the discussion in [ref to GAUSS-CODAZZI] about co-dimension one hyper-surfaces, we will fix $D=5$. This implies that the boundary $\partial B$ of the nucleated five-dimensional bubble will have $d = D-1 = 4$ dimensions.

In this specific scenario, the geometry inside and outside the bubble can be described by the following line invariant:
\begin{equation}
	ds_{\pm}^{2} = g^{\pm}_{\mu\nu} dx^{\mu} dx^{\nu} =  -f_{\pm}(r) dt_{\pm}^{2} + f^{-1}_{\pm}(r) dr^{2} + r^{2} d\Omega_{3}^{2},
\end{equation}
where
\begin{equation}\label{eq: vacuum_func}
	f_{\pm}(r) = 1 + k_{\pm}^{2} r^{2} + \chi(r, k_{\pm}, q_{1},..., q_{m}).
\end{equation}
Here, $r$ will denote the radial coordinate of the $\text{AdS}_{5}$ vacuum, i.e. the throat direction. The line-invariant $d\Omega_{3}^{2} = \gamma_{ij}dx^{i}dx^{j}$ is the metric on $S^{3}$, which corresponds to the three-dimensional solid angle for the usual spatial sections. The function $f_{\pm}(r)$ represents the usual $\text{AdS}_{5}$ geometry plus some posible extra features encoded in $\chi(r, k_{\pm}, q_{1},..., q_{m})$. These will be relevant in the following sections, but for now on, let us fix $\chi(r, k_{\pm}, q_{1},..., q_{m}) =0$.

CONTINUE EXPLANAINING PARAMETRISATION OF FOUR D AND STUFF.





\section{Dark bubbles from string theory: An explicit construction}
Here some general shity introduction about the general features of the model in 10D.
\subsection{A rotating ten-dimensional black hole}
Talk about whole construction that Oscar used and present all ingredients in the game. Two option: Describe the basics of the dual field theory here or create appendix about AdS/Cft?
\subsection{Comparing junctions to EOM}
Basically, compare as we did in the paper. Just to prove, that at some specific limit EOM = JC. 
\subsection{Higher curvature corrections to the rescue}
Point to the fact that previous EOM has no CC. Discuss about how to obtain it and elaborate on all pieces in the game. End with the famous $1/N$ corrections that transform in CC. (First part of 5th paper)
\subsection{Energy scales from Dark Bubble embedding: A new hope}
Compare to observed CC and obtain $N$ to fix all scales. Discuss and defend the size of extra dimensions and the stringy scales.
\section{Decorating the cosmos: Gauss-Codazzi equation at work}
Argue somehow that we are going to look at things from a 5D perspective and will start to induce stuff in the 4D cosmos.
\subsection{Matter}
\subsection{Radiation}
\subsection{Gravitational waves}
\subsection{The standard model of particle physics}
\subsubsection{Electromagnetism}
\subsubsection{Weak force: Neutrinos}
\section{Holographic bubbles and hanging strings}
\section{Quantum bubbles: Higher dimensions to solve boundary choices}

\cite{Danielsson:2018ztv}




\bibliography{biblio}
\bibliographystyle{JHEP}


\end{document}