\documentclass[12pt, a4paper]{article} % Font size
\usepackage{float} % To impose the position of the figure at desired place.
\usepackage{amsmath, amsfonts, amsthm} % Math packages
\usepackage{hyperref} % To hiperlink things on the internet.
\usepackage{cite} % To cite stuff
\usepackage{listings} % Code listings, with syntax highlighting
\usepackage[english]{babel} % English language hyphenation
\usepackage{graphicx} % Required for inserting images
\usepackage{sectsty} % Allows customising section commands
\usepackage{booktabs} % Required for better horizontal rules in tables
\usepackage{enumitem} % Required for list customisation
\usepackage{geometry} % Required for adjusting page dimensions and margins
\usepackage{svgcolor}
\usepackage{svg}
\usepackage{multirow}
\usepackage{comment} % To remove big parts of text.
\usepackage[utf8]{inputenc} % Required for inputting international characters
\usepackage[T1]{fontenc} % Use 8-bit encoding
\usepackage{fourier} % Use the Adobe Utopia font for the document
\usepackage[nottoc]{tocbibind}
\usepackage{cancel} % To cross and cancel values
\usepackage{bigints} % In case you need bigger integrants
\usepackage{xcolor} %To use colors
\usepackage{subcaption}
\usepackage[skip=10pt plus1pt, indent=0pt]{parskip}
\newcommand{\ket}[1]{\vert #1 \rangle}
\newcommand{\bra}[1]{\langle #1 \vert}
\renewcommand{\it}{\textit}
\definecolor{Burgundy}{RGB}{155,5,8}
\hypersetup{
    colorlinks=true,
    linkcolor= Burgundy,
	citecolor = Burgundy,
    filecolor=magenta,      
    urlcolor=Burgundy,
    }



\geometry{
	paper=a4paper, % Paper size, change to letterpaper for US letter size
	top=2.5cm, % Top margin
	bottom=2.5cm, % Bottom margin
	left=2.5cm, % Left margin
	right=2.5cm, % Right margin
	headheight=0.5cm, % Header height
	footskip=1.5cm, % Space from the bottom margin to the baseline of the footer
	headsep=0.75cm, % Space from the top margin to the baseline of the header
	%showframe, % Uncomment to show how the type block is set on the page
}




\setlength\parindent{0pt} % Removes all indentation from paragraphs
\graphicspath{{Figures/}{./}}% Specifies where to look for included images (trailing slash required)
%\setlist{noitemsep} % No spacing between list items
\sectionfont{\vspace{6pt}\centering\normalfont\scshape} % \section{} styling
\subsectionfont{\normalfont\bfseries} % \subsection{} styling
\subsubsectionfont{\normalfont\itshape} % \subsubsection{} styling
\paragraphfont{\normalfont\scshape} % \paragraph{} styling










%--------------------------------------------------------------------------------
%	TITLE SECTION
%--------------------------------------------------------------------------------

\title{	
	\normalfont\normalsize
	\textsc{}\\ % Your university, school and/or department name(s)
	\vspace{5pt} % Whitespace
	\rule{\linewidth}{0.2pt}\\ % Thin top horizontal rule
	\vspace{10pt} % Whitespace
	{\huge About Dark Bubbles}\\ % The assignment title
	\vspace{5pt} % Whitespace
	\rule{\linewidth}{-2pt}\\ % Thick bottom horizontal rule
	\vspace{-25pt} % Whitespace
	\date{}
}
\author{}





\begin{document}
\maketitle 
\tableofcontents



	%\begin{figure}[ht]\label{fig: something}
     %   \centering
      %  \includegraphics[scale=0.7]{test}
       % \caption{caption}
      %\end{figure}

\newpage
\section{How to induce dark energy on a bubble: A simple example}

Let us just start with the most basic set-up: A non-supersymmetric Anti-de Sitter vacuum, with cosmological constant $\Lambda_{D} = - \tfrac{1}{2}(D-1)(D-2)L^{3-D}$. The parameter $L$ represents the curvature radius of the $\text{AdS}_{D}$ vacuum. For computational convenience, it will be easier to refer to this parameter with its inverse, the vacuum scale $k = \tfrac{1}{L}$.

As we discussed in section [NON-SUSY], the aforementioned $\text{AdS}_{D}$ vacuum is unstable and will eventually decay to a more metastable solution, via $D_{p}$-brane nucleation. This $D_{p}$-brane, which is a Coleman-de Luccia/ BT instanton [add ref to CdL and or section], will mediate such decay, separating two different $\text{AdS}_{D}$ vacua. From now on, we will refer to the vacua living inside the hyper-volume encoded by the $D_{p}$-brane as the \textit{inside} vacuum, denoted by a minus (-) subscript sign. The whole higher-dimensional space receives the name of \textit{Bulk} and the tandem brane-inside will be colloquially called \textit{Bubble}. On the other hand, the region \textit{outside} the bubble has not yet decayed, and will be denoted by a positive (+) subscript sign.

\textcolor{red}{I THINK THAT A SKETCH WITH NAMES AND K VALUES HERE COULD BE NICE.}

Finally, as the aim of this thesis is to discuss phenomenological aspects in four dimensions\footnote{Other bubble dimensionalities have been explored in [IVANO STUFF]}, and given the discussion in [ref to GAUSS-CODAZZI] about co-dimension one hyper-surfaces, we will fix $D=5$. This implies that the boundary $\partial B$ of the nucleated five-dimensional bubble will have $d = D-1 = 4$ dimensions.

In this dimensionality, the simplest geometry inside and outside the bubble can be described by the following line invariant:
\begin{equation}\label{eq: basic_global_bubble}
	ds_{\pm}^{2} = g^{\pm}_{\mu\nu} dx^{\mu} dx^{\nu} =  -f_{\pm}(r) dt_{\pm}^{2} + f^{-1}_{\pm}(r) dr^{2} + r^{2} d\Omega_{3}^{2},
\end{equation}
where
\begin{equation}\label{eq: vacuum_func}
	f_{\pm}(r) = 1 + k_{\pm}^{2} r^{2} + \chi(r, t, k_{\pm}, q_{1},..., q_{m}).
\end{equation}
Here, $r$ will denote the radial coordinate of the $\text{AdS}_{5}$ vacuum, i.e. the throat direction and the line-invariant $d\Omega_{3}^{2} = \gamma_{ij}dx^{i}dx^{j}$ is the metric on $S^{3}$, which corresponds to the three-dimensional solid angle for the usual spatial sections. The function $f_{\pm}(r)$ represents the $\text{AdS}_{5}$ geometry plus some posible extra features encoded in $\chi(r, t, k_{\pm}, q_{1},..., q_{m})$. These will be relevant in the following sections, but for now on, let us fix $\chi(r, t, k_{\pm}, q_{1},..., q_{m}) =0$.

Before we start computing the Israel's junction conditions described in [ref to APPENDIX], it is important to realise the parametrical dependence of the expanding bubble. As described in [ref to INSTANTONS], the nucleated bubble can be identified with a Coleman-de Luccia instanton, which is equipped with an $O(4)$-symmetry.\footnote{Both in Euclidean and Lorentzian time by analytic continuation.} The two relevant coordinates for this derivation are the global time $t$ and the radial coordinate $r$, which also depends on $t$, as it is expanding, hence evolving in time. Both will be related to the proper time $\tau$ an observer living on the boundary $\partial B$ experiments. The coordinate choices to describe the bulk geometry and the induced one are:
\begin{equation}\label{eq: coordinates}
	x^{\mu} = \{t(\tau), r(\tau), \theta, \phi, \psi \}, \qquad \qquad  y^{a} = \{\tau, \theta, \phi, \psi\}.
\end{equation}
Let us now compute the first Israel junction condition described in [ref to APPENDIX]. This is no more than:
\begin{equation}\label{eq: induced_FRW}
	ds^{2}_{\text{ind}} = h_{ab} dy^{a} dy^{b} =  -N(\tau)^{2}d\tau^{2} + a^{2}(\tau) d\Omega_{3}^{2},
\end{equation}
where
\begin{equation}\label{eq: lapse_func}
	N^{2} = f(a(\tau)) \dot{t}^{2} - \frac{\dot{a}(\tau)^{2}}{f(a(\tau))}.
\end{equation}
The metric $h_{ab}$ represents the four-dimensional induced geometry on the hyper-surface, i.e. the $D_{3}$-brane, given by the parametrisation of $\{y^{a}\}$-coordinates. The lapse function $N(\tau)$ has been introduced to make time reparametrisation invariance manifest. Observe that a dot represents a proper-time derivative. One can always choose the right parametrical relation between $f(a(\tau)), \dot{t}(\tau)$ and $a(\tau)$ such that $N=1$. In that case, it is always easy to see that (\ref{eq: induced_FRW}) represents an expanding four-dimensional cosmology, described by a Friedmann-Robertson-Lemaître-Walker metric with closed spatial sections. 

What about the second Israel junction condition? As discussed in [APPENDIX], this condition relates how the brane is embedded in the bulk geometry to the energy-momentum tensor induced on the brane, such that the whole bulk-brane geometry remains a solution to the Einstein equation. For simplicity in this example, let us assume we have an empty brane. This implies that its energy-momenum tensor only contains information about its tension $\sigma$, which can be denoted by:
\begin{equation}
	S_{ab} = - \sigma h_{ab}.
\end{equation}
\textcolor{red}{COMMENT ON THE MINUS SIGN? ChECK CRITICAL SIGMA VALUE FOR POINCARE}

As discussed in [APPENDIX], the second Israel's condition relates the \textit{jump} in the extrinsic curvature between the inside and the outside geometries that the brane separates. To be as pedagogical as possible in this introductory section, we will compute the most important steps to obtain the induced energy-momentum tensor.\footnote{The whole derivation is left as an exercise to the reader.}

First, we need to identify the normal and tangent vectors $\{n^{\mu}, e^{\mu}_{a}\}$ with respect to the embedding. Using eqs (APPENDIX) and (APPENDIX),  these result to be:
\begin{small}
\begin{equation}
	e^{\mu}_{a} = \begin{pmatrix} \dot{t}(\tau) &0&0&0&0\\\dot{f}(a)&0&0&0&0\\0&0&1&0&0\\0&0&0&1&0\\0&0&0&0&1 \end{pmatrix}, \qquad n^{\mu} = \left(\frac{-\dot{a}/\dot{t}}{f(a)\sqrt{f(a) - \tfrac{1}{f(a)} \left(\tfrac{\dot{a}}{\dot{t}}\right)^{2}}}, \frac{f(a)}{\sqrt{f(a) - \tfrac{1}{f(a)} \left(\tfrac{\dot{a}}{\dot{t}}\right)^{2}}}, 0,0,0 \right).
\end{equation}
\end{small}
This implies an extrensic curvature $K_{ab}$ of the form:
\begin{equation}
	K_{ab} = \nabla_{\beta} n_{\alpha} e^{\alpha}_{a} e^{\beta}_{b} = 
\end{equation}
\textcolor{red}{CONTINUE HERE.}




\section{Dark bubbles from string theory: An explicit construction}
Here some general shity introduction about the general features of the model in 10D.
\subsection{A rotating ten-dimensional black hole}
Talk about whole construction that Oscar used and present all ingredients in the game. Two option: Describe the basics of the dual field theory here or create appendix about AdS/Cft?
\subsection{Comparing junctions to EOM}
Basically, compare as we did in the paper. Just to prove, that at some specific limit EOM = JC. 
\subsection{Higher curvature corrections to the rescue}
Point to the fact that previous EOM has no CC. Discuss about how to obtain it and elaborate on all pieces in the game. End with the famous $1/N$ corrections that transform in CC. (First part of 5th paper)
\subsection{Energy scales from Dark Bubble embedding: A new hope}
Compare to observed CC and obtain $N$ to fix all scales. Discuss and defend the size of extra dimensions and the stringy scales.
\section{Decorating the cosmos: Gauss-Codazzi equation at work}
Argue somehow that we are going to look at things from a 5D perspective and will start to induce stuff in the 4D cosmos.
\subsection{Matter}
\subsection{Radiation}
\subsection{Gravitational waves}
\subsection{The standard model of particle physics}
\subsubsection{Electromagnetism}
\subsubsection{Weak force: Neutrinos}
\section{Holographic bubbles and hanging strings}
\section{Quantum bubbles: Higher dimensions to solve boundary choices}

\cite{Danielsson:2018ztv}




\bibliography{biblio}
\bibliographystyle{JHEP}


\end{document}